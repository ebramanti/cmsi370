\documentclass[11pt]{article}
\usepackage{geometry}                % See geometry.pdf to learn the layout options. There are lots.
\geometry{letterpaper}                   % ... or a4paper or a5paper or ... 
%\geometry{landscape}                % Activate for for rotated page geometry
%\usepackage[parfill]{parskip}    % Activate to begin paragraphs with an empty line rather than an indent
\usepackage{graphicx}
\usepackage{amssymb}
\usepackage{epstopdf}
\usepackage{indentfirst}
\DeclareGraphicsRule{.tif}{png}{.png}{`convert #1 `dirname #1`/`basename #1 .tif`.png}

\title{Small Screen, Big Strategy: Bringing Real-Time Strategy to Mobile Devices}
\author{Edward Bramanti}

\begin{document}
\maketitle
\pagebreak
\section{Introduction}
In the past few years, mobile operating systems have become a hub for both independent developers and game studios alike to publish video games that their consumers can hold in the palm of their hand. Mobile game development once used to be a frontier; now, it seems as though a new game comes out every few weeks, and they battle for the addiction of their consumers. While many different genres of mobile games have been produced, from tower defense to adventure to even massively multiplayer, real-time strategy remains a genre mostly untouched by mobile game developers. One's initial reaction may be to believe that lack of interest in real-time strategy (or RTS) games is responsible for few games being published in that genre. Nevertheless, as this design will outline, a lack of intuitive controls and user interface has prevented the genre from thriving on mobile. Many developers have been met with interface design challenges. One example of this is a result of lacking hardware devices outside of the touchscreen (such as a mouse and keyboard, or even a controller with buttons). My user design will show how mobile strategy is the future of challenging, yet rewarding mobile gaming. Empowering the users by utilizing every inch of screen is the key foundation to building a successful mobile RTS interface.
\section{Design \& Layout}
\section{Usage Scenarios}
\section{Design Rationale}
\section{Usability Analysis}
\end{document}  