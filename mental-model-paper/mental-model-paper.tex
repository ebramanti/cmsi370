\documentclass[11pt]{article}
\usepackage{geometry}                % See geometry.pdf to learn the layout options. There are lots.
\geometry{letterpaper}                   % ... or a4paper or a5paper or ... 
%\geometry{landscape}                % Activate for for rotated page geometry
%\usepackage[parfill]{parskip}    % Activate to begin paragraphs with an empty line rather than an indent
\usepackage{graphicx}
\usepackage{amssymb}
\usepackage{epstopdf}
\usepackage{indentfirst}
\DeclareGraphicsRule{.tif}{png}{.png}{`convert #1 `dirname #1`/`basename #1 .tif`.png}

\title{Evaluating the Mobile Operating System Effectively}
\author{Edward Bramanti}
%\date{}                                           % Activate to display a given date or no date

\begin{document}
\maketitle
\begin{abstract}
\end{abstract}
\pagebreak
\section{Introduction}
In today's world, mobile operating systems are now as prevalent as personal computers. More and more, consumers are turning to smartphones and tablets to daily drive their time spent in productivity as well as entertainment. As this field has continued to develop, the consumer is given a myriad of choices between many different operating systems. Hardware and software manufacturers have arisen to provide a competitive environment. This has led to successes for user choice, as hardware and software companies square off to provide the most innovative interface.\\
\indent While this has been a great success for the consumer, it has also caused problems for firms. A series of patent wars between companies has arisen in the past few years, which has demonstrated the desire for monopoly over the market by utilizing the legal system. Paik and Zhu note that "According to this view, technology firms race to assemble patent portfolios - initially for defensive purposes in the context of a dynamic and competitive field - but then, as the industry matures, convert their shields into weapons to eliminate their competitors in pursuit of market dominance with its platform."\cite{PatentWars} This desire to achieve market independence has led many reviewers to create preconceived notions that the market is no longer innovating, but instead creating a "future-proofing" system that fights pointless legal battles to control software innovation.\\
\indent Because of this new growth in the smartphone market, reviewers are pitting operating system against operating system to help the user discover if it is the best fit for them. The question remains though: Are the journalistic and firm-based reviewers doing an accurate job of evaluating a mobile operating system? In particular, we will be taking a look at Pfeiffer Consulting's "Mobile OS User Experience Shootout" as a reference for how companies conduct evaluations of mobile operating systems today, and how focusing on certain things that connect consumers and developers together through mental process will increase reasoning and understanding behind the results collected from users.
\section{Background}
To provide background on how evaluations of a mobile operating system should be performed, we will look at the data from the two most important points of view: first, the developer; and then, the consumer. Seeing how the developer analyzes the operating system versus the end user will help to provide a bridge between the mental thought process of the two, and will provide those observing a foundation for how to approach reviewing a mobile operating system effectively.
\subsection{The Mobile OS - Developer Perspective}
\section{Methods}
\section{Discussion}
\section{Conclusion}

\bibliographystyle{alpha}
\bibliography{mental-model-paper}

\end{document}  